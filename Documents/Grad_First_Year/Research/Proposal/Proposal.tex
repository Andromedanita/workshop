\documentclass[preprint]{aastex}
\usepackage{mathtools}
\usepackage{authblk}
\usepackage{natbib}
%\usepackage[top=0.8in, bottom=0.6in, left=1in, right=1in]{geometry}

\bibliographystyle{apj}

\title{Dynamics of Stellar Streams to constrain Milky Way potential}
\affil{\small {Anita Bahmanyar}}
%\affil{\small {bahmanyar@astro.utoronto.ca}}
\affil{\small {Supervised by: Prof. Jo Bovy}}
%\affil{\small {Department of Astronomy and Astrophysics, University of Toronto}}
\date{}

\newcommand{\kpc}{\mathrm{kpc}}
\newcommand{\kms}{\mathrm{kms^{-1}}}



\begin{document}


\begin{abstract}
In this project, we will be looking at stellar streams generated by the tidal disruption of stars from a satellite galaxy or a globular cluster.
We fit an orbit to the stream using single and multi-component potentials that best describe the stream orbit. We will use Grillmair Dionatos (GD-1) stream data obtained from SDSS photometry and Calar Alto spectroscopy. Once we have fitting parameters, we look at the likelihood- the probability of occurrence of the fitted orbital parameters. These include stream position, proper motion and its distance from centre of the Milky Way. We then optimize the parameters to get the largest likelihood value. Previously, it was assumed that if the stream is thin, an orbit can be fitted to the stream; however, this causes bias in constraining the potential using the stream, as this assumption is not  very accurate. This is because the stars at the ends of the stream have larger frequency differences from the progenitor than that of the stars closer to the progenitor.
\end{abstract}

\maketitle

\section{Introduction}
Stars in satellite galaxies and clusters get tidally disrupted by their host galaxy as they orbit around it. The orbit of the tidally disrupted stars is close to that of their progenitor, extending ahead and beyond, creating leading and trailing tails, respectively \citep{Bowden2015}. Understanding the physics of these stream orbits help us study the structure of the host galaxy, the shape of galactic halo and allows us to constrain galaxy's potential. Tidal streams can also give us information about the large- and small-scale structure of the Milky Way halo's density distribution \citep{Bovy2014}.

A number of these streams have been detected within our own Milky Way galaxy, which could help us constrain the potential of the Milky Way. The most famous example is the Sagittarius (Sgr) dwarf galaxy, discovered in 1994 \citep{Ibata1994}. The nucleus of Sgr has survived for many orbits around the Milky Way, while its tidal tails have now been detected over a full $360^{\circ}$ on the sky and provides a strong constraint on the Milky Way's halo \citep{Fellhauer2006}. 
Some of the other detected streams in the Milky Way galaxy are GD-1 stream, Orphan stream and NGC5466 stream. 
These streams are derived from progenitors with lower mass than that of the Sgr stream \citep{Bowden2015}.

The difficulty in modelling the streams is that they do not follow a single orbit. We can fit more than one single orbit to the streams but it would be computationally expensive, which has led to the assumption of fitting a single orbit to the streams \citep{Bovy2014}. \citet{koposov} mention that if the stream is very thin, we can make the assumption that the stream stars move along the same orbit even though in general the stars have different energies and angular momenta, but \citet{Sanders2013} found that this can lead to systematic biases in estimates of the potential parameters. This is due to the larger difference in frequencies of the stars at the end parts of the stream from their progenitor than that of the stars closer to the progenitor.

In this project, I will be using \texttt{galpy}, a Python package written by Jo Bovy \citep{galpy2015} for galactic dynamics calculations. The \texttt{galpy} package fits for different galactic potentials using a variety of integration methods. The units in \texttt{galpy} are in natural units: the circular velocity is one at a cylindrical radius of one and height of zero \citep{galpy2015}. One needs to multiply the output by the actual values to convert to physical units. For instance, position and velocity should be multiplied by $8.5 \, \kpc$ and $220 \, \kms$, respectively  in a model where the Sun is assumed to be at a distance $8.5 \, \kpc $ from the Galactic centre and has the circular velocity of $220 \, \kms$.


%A large number of dark matter sub halos reside within the Milky Way galaxy. Since dark matter does not interact with photos, we need indirect ways to confirm its existence. Using streams would help us study these sub halos. Occasionally, there are mysterious gaps in the stellar streams that could indicate existence of dark matter sub halos when the stream passes behind them.  \citep{Bovy2014}

\section{Data}
GD-1, first detected by \citet{GD2006}, is a cold thin stream that is $63^{\circ}$ long on the sky. It is suggested that it was generated from a globular cluster since it is very thin, but there is no progenitor remnant to confirm this hypothesis \citep{Sanders2013}. It is located at $\sim 8.5 \, \kpc$ from the Sun radially and $ \sim15 \kpc$ from the Galactic centre and is moving perpendicular to the line of sight with the velocity of 220 $\kms$ \citep{koposov}.
We will be using data on the GD-1 stellar stream, which is a combination of the Sloan Digital Sky Survey (SDSS) photometry and Calar Alto spectroscopy. These data include position of the stream, radial velocity, proper motion and distance of the stream, given in the tables 1-4 in \citet{koposov}. The stream positions are given in stream coordinates $\phi_1$ and $\phi_2$ which is a rotated coordinate system approximately aligned with the stream. $\phi_1$ and $\phi_2$ represent the longitude and latitude in the stream, respectively. The proper motions are also given in stream coordinates which can easily be converted to Galactic coordinates. 


\section{Method} \label{label:methodsection}

\subsection{Orbit Fitting}
To begin, we assume a single-component potential known as flattened logarithmic potential for the Milky Way. This potential is given by:

\begin{equation} \label{eq:logPotential}
\Phi(x,y,z) = \frac{V_c^2}{2} \, \mathrm{ln} \left ({x}^2 + y^2 + \left ( \frac{z}{q_{\Phi}}\right )^2 \right ),
\end{equation}
where $V_c$ represents the circular velocity and $q_{\Phi}$ is the flattening parameter. 

We need the initial position and velocity components of the stream as well as the distance to the stream. Once we have the orbit initialized, we can integrate it to get the orbital properties such as the position of stream at any given time. An application of fitting an orbit to the GD-1 stream data is shown in Figure (1), where we have used initial conditions of the GD-1 stream given in \citet{koposov} to predict its orbit at later times.


\begin{comment}
From orbit integration, we get the proper motion of the stream in Galactic coordinates ($l,b,d$). We also want to have the proper motion of the stream in the stream coordinates ($\mu_{\phi_1},\mu_{\phi_2},V_{los}$). This transformation can be done using equation (\ref{eq:proper_lb_phi12}):

\begin{equation}  \label{eq:proper_lb_phi12}
\begin{bmatrix}
\mu_r
\\ \mu_{\phi_1}
\\ \mu_{\phi_2}
\end{bmatrix} = \mathrm{R \,T \, A} \begin{bmatrix}
\mu_r
\\ \mu_{l}
\\ \mu_{b}
\end{bmatrix},
\end{equation}

where the matrices R, T and A are defined below by equations (\ref{eq:R}), (\ref{eq:T}) and (\ref{eq:A}), respectively:

\begin{equation} \label{eq:R}
\mathrm{R} = 
\begin{bmatrix}
 \cos(\phi_1) \cos(\phi_2) & \cos(\phi_2) \sin(\phi_1)  & \sin(\phi_2) \\ 
-\sin(\phi_!) &  \cos(\phi_1) & 0  \\
-\cos(\phi_1) \sin(\phi_2)  &  -\sin(\phi_1) \sin(\phi_2) & \cos(\phi_2) 
\end{bmatrix}
\end{equation}


\begin{equation} \label{eq:T}
\mathrm{T} = 
\begin{bmatrix}
-0.4776303088  &  0.510844589 & 0.7147776536 \\ 
-0.1738432154 & -0.8524449229 & 0.493068392  \\
 0.8611897727  &  0.111245042 & 0.4959603976 
\end{bmatrix}
\end{equation}


\begin{equation} \label{eq:A}
\mathrm{A} = 
\begin{bmatrix}
 \cos(\alpha) \cos(\delta) & -\sin(\alpha) & -\cos(\alpha) \sin(\delta) \\ 
\sin(\alpha) \cos(\delta) & \cos(\alpha) & -\sin(\alpha) \sin(\delta)\\
\sin(\delta)  &  0 & \cos(\delta) 
\end{bmatrix}
\end{equation}
\end{comment}



As mentioned before, fitting a single orbit cannot be very accurate for obtaining the potential of the Milky Way, since stars orbit on different paths from one another. In this case, stream models create a better understanding of the stream orbit.

%We can convert the stream coordinates to equatorial coordinates, $\alpha$ and $\delta$ using equation (\ref{eq:phi12}):

%\begin{equation} \label{eq:phi12}
%\begin{bmatrix}
%\cos(\phi_1) \cos(\phi_2)
%\\ \sin(\phi_1) \cos(\phi_2) 
%\\ \sin(\phi_2)
%\end{bmatrix} = \mathrm{T}
%\begin{bmatrix}
%\cos(\alpha) \cos(\delta)
%\\\sin(\alpha)\cos(\delta)
%\\ \sin(\delta),
%\end{bmatrix}
%\end{equation}

\subsection{Likelihood}
Likelihood is the probability of getting a y-value at an x-value given a model. In our case, we will consider likelihood as the probability of getting $\phi_2(t)$, distance(t), $V_{\mathrm{rad}}(t)$  or $\mu(t)$ at a specific $\phi_1$ given a model, which corresponds to  $\mathcal{L} \propto \mathrm{P}((\phi_2, \mathrm{d}, V_{\mathrm{rad}} \, \mathrm{or} \, \mu) \, \mathrm{at} \, \phi_1 | \mathrm{model \, at \, time \, t})$.
The logarithmic likelihood for the case of $\phi_2$ can be written as in equation (2), where an approximate form of it is followed in the next line. We need to integrate over time to get an average value since we do not know the time the data has been taken. But since we do not have an infinite number of times, we will only take the sum.
\begin{eqnarray} \label{eq:likelihood}
\mathcal{L}  & \propto & \int \exp^{\frac{-(\phi_1(t) - {\phi_1^{obs}})^2}{2\sigma_1^2} + \frac{-(\phi_2(t) - {\phi_2^{obs}})^2}{2\sigma_2^2}} dt  \\ \nonumber
\mathcal{L}  & \propto & \sum_i \exp^{\frac{-(\phi_1(t) - {\phi_1^{obs}})^2}{2\sigma_1^2} + \frac{-(\phi_2(t) - {\phi_2^{obs}})^2}{2\sigma_2^2}}.  \nonumber \\ \nonumber
\end{eqnarray}
We can write the same expression as in equation (\ref{eq:likelihood}) for $\mathrm{d}, V_{rad}$ and $\mu$.  This is shown for $V_{rad}$ in equation (\ref{eq:likelihoodsum}):
\begin{equation} \label{eq:likelihoodsum}
\mathcal{L}  & \propto & \sum_i \exp^{\frac{-(\phi_1(t) - {\phi_1^{\mathrm{obs}}})^2}{2\sigma_1^2} + \frac{-(V_{\mathrm{rad}}(t) - {V_{\mathrm{rad}}^{\mathrm{obs}}})^2}{2\sigma_2^2}}. 
\end{equation}
In general we can write:
\begin{equation} \label{eq:chi2}
\ln (\mathcal{L}) & = & - \frac{\chi^2}{2 } =  \prod_i \frac{(x_{\mathrm{model},i}-x_{\mathrm{data},i})^2}{2\sigma_i^2},
\end{equation}
where $i$ represents each of the data points and $\sigma_i$ is the associated error. This means that we need to multiply the logarithmic likelihood value of each point to get a total likelihood value. An application of this is shown in Figure (1).

\newpage
\section{Timeline}
\begin{itemize}
\item Fitting orbit to the GD-1 stream and calculating parameter likelihood by the end of November
\item Fit stream model for fixed a potential by the end of December
\item Fit stream model for a varying potential by the end of January
\item Look at different potential families by the end of February
\item Wrapping up and writing final report by the end of March
\end{itemize}

\newpage

\FloatBarrier
\begin{figure}[H] \label{label:fig1}
\centering
\includegraphics[scale=0.5]{Table2.png}
\caption{\textit{Left:} Represents an application of orbit fitting explained in section \ref{label:methodsection} and it shows the values of $\phi_2$ vs. $\phi_1$. The red dots represent the GD-1 stream data as in \citet{koposov} tables along with the error bars on $\phi_2$. The error bars on the $\phi_1$ values are all the same and $1 ^\circ{}$. The black curve is the fitted orbit to the stream using \texttt{galpy} and the stream initial conditions. \textit{Right:} Represents the $\chi^2$ value obtained from calculating the log-likelihood of the $\phi_2$ and $\phi_1$ values using equation (\ref{eq:chi2}).}
\end{figure}
\FloatBarrier

\bibliography{Proposal}

\end{document}




